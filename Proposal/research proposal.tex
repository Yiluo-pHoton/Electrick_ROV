\documentclass[a4paper,12pt]{article}
\usepackage[utf8]{inputenc}
\usepackage{cite}
\usepackage{indentfirst}
\usepackage{graphicx}
\usepackage{dblfloatfix}
\usepackage{authblk}

\newcommand*{\Titlefont}{\Large\normalfont}
\renewcommand*{\Authfont}{\normalsize\normalfont}
\renewcommand*{\Affilfont}{\footnotesize\normalfont}

%%%%%%%%%%%%%%%% TABLE
\setlength{\arrayrulewidth}{0.1mm}
\setlength{\tabcolsep}{18pt}
\renewcommand{\arraystretch}{1.5}
\newcommand{\ce}{\centering}

%%%%%%%%%%%%%%%%% HEADING
\title{\Titlefont PHYS CS 15C Research Proposal\\Remotely Operated Vehicle with Visualized Terrain}
\author[1]{\Authfont David Wang}
\author[1]{\Authfont Frank Fu}
\author[1]{\Authfont Yiluo Li}
\affil[1]{\Affilfont College of Creative Studies, University of California, Santa Barbara}

\begin{document}

\maketitle

\section{Introduction}
Often times there exist terrain on which people cannot tread on. Imagine the complex structure of the debris after an earthquake or hurricane, where an external force of a human stepping on it might cause additional collapse of the structure, putting victims under it in further danger. However, we would like to conduct massive search for survivors under the debris. General search could be done by some high-end device far away, and close up confirmation for each potential signal of life could be carried out by smaller sized vehicles such as drones and ground robots. While drones have high mobility around, the ground robot can go under small holes closer to the survivors. By carrying necessary communication tools and sensors, we could get specific conditions of the survivors, which would be immensely helpful in forming the rescue plan accordingly.
%%%%%%%%%%%%%
\\\indent However, even when a signal of life has been detected, there are still obstacles. Removing the top layers of debris might do damage to the lower layer. Depending on the actual situation of the victim and the structure above the victim, a mature rescue plan might take hours in finalizing. Time is an important factor in this process. To earn more time and make sure the victim can stay with us, a ground robot that have access to the victim could then provide necessary care such as food, water, conversation, and hope to stabilize the conditions of the victim.


\section{Significance}



\section{Objective}

\begin{enumerate}
	\item Study the differences and similarities of three icy satellites, and propose different modifications to be made for the current Europa model;
	\item Extend the Europa model\cite{Trumbo2017} to fit to Galileo PPR data and ALMA observations for Ganymede and Callisto; 
	\item Analyze the temperate maps with the developed models and potentially identify other similarly anomalous regions.
\end{enumerate}


\section{Methodology}



\section{Proposed Project Timeline}

$$
\begin{tabular}{ |p{1cm}|p{10cm}|  }
	\hline
		\multicolumn{2}{|c|}{Research Timeline} \\
	\hline
		 Week & Description \\
	\hline
		\ce 1		& 	Order necessary parts\\
		\ce 2		&	Assemble the conductive layer control pad, supply constant AC current through one pair of electrodes, and measure the voltage difference at different vertices\\
		\ce 3-4		&	Supply current through all adjacent pairs of electrodes, and readout the voltage differences at all other vertices\\
		\ce 5-6		&	Visualize the 2-D voltage current density	when touching the control pad with tomography imaging\\
		\ce 7		&	Output coordinate of touch on the control pad; assemble the self-balancing robot\\
		\ce 8-9		&	Assemble the self-balancing robot and the bluetooth module\\
		\ce 10		&	Move the robot with control robot\\
	\hline
\end{tabular}
$$

\section{Conclusion}


\section{Acknowledgment}
I would like to acknowledge Qicheng Zhang and my advisor at UCSB Dr. Tengiz Biblilashvli for their valuable feedback toward this research proposal.

\medskip

%%%%%%%%%%%%%%%%%%%%%%%%%%%%%%%%%%%%%%% CITATION
\bibliographystyle{phaip}
\bibliography{ref}

\end{document}
