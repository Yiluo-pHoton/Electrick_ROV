\documentclass[a4paper,12pt]{article}
\usepackage[utf8]{inputenc}
\usepackage{cite}
\usepackage{indentfirst}
\usepackage{graphicx}
\usepackage{dblfloatfix}
\usepackage{authblk}

\newcommand*{\Titlefont}{\Large\normalfont}
\renewcommand*{\Authfont}{\normalsize\normalfont}
\renewcommand*{\Affilfont}{\footnotesize\normalfont}

%%%%%%%%%%%%%%%% TABLE
\setlength{\arrayrulewidth}{0.1mm}
\setlength{\tabcolsep}{18pt}
\renewcommand{\arraystretch}{1.5}
\newcommand{\ce}{\centering}

%%%%%%%%%%%%%%%%% HEADING
\title{\Titlefont PHYS CS 15C Research Proposal\\Remotely Operated Vehicle with Visualized Terrain}
\author[1]{\Authfont David Wang}
\author[1]{\Authfont Frank Fu}
\author[1]{\Authfont Yiluo Li}
\affil[1]{\Affilfont College of Creative Studies, University of California, Santa Barbara}

\begin{document}

\maketitle

\section{Introduction}



\section{Significance}



\section{Objective}

\begin{enumerate}
	\item Study the differences and similarities of three icy satellites, and propose different modifications to be made for the current Europa model;
	\item Extend the Europa model\cite{Trumbo2017} to fit to Galileo PPR data and ALMA observations for Ganymede and Callisto; 
	\item Analyze the temperate maps with the developed models and potentially identify other similarly anomalous regions.
\end{enumerate}


\section{Methodology}



\section{Proposed Project Timeline}

$$
\begin{tabular}{ |p{1cm}|p{10cm}|  }
	\hline
		\multicolumn{2}{|c|}{Research Timeline} \\
	\hline
		 Week & Description \\
	\hline
		\ce 1		& 	Order necessary parts\\
		\ce 2		&	Assemble the conductive layer control pad, supply constant AC current through one pair of electrodes, and measure the voltage difference at different vertices\\
		\ce 3-4		&	Supply current through all adjacent pairs of electrodes, and readout the voltage differences at all other vertices\\
		\ce 5-6		&	Visualize the 2-D voltage current density	when touching the control pad with tomography imaging\\
		\ce 7		&	Output coordinate of touch on the control pad; assemble the self-balancing robot\\
		\ce 8-9		&	Assemble the self-balancing robot and the bluetooth module\\
		\ce 10		&	Move the robot with control robot\\
	\hline
\end{tabular}
$$

\section{Conclusion}


\section{Acknowledgment}
I would like to acknowledge Qicheng Zhang and my advisor at UCSB Dr. Tengiz Biblilashvli for their valuable feedback toward this research proposal.

\medskip

%%%%%%%%%%%%%%%%%%%%%%%%%%%%%%%%%%%%%%% CITATION
\bibliographystyle{phaip}
\bibliography{ref}

\end{document}
